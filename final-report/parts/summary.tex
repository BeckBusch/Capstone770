%----------------------------------------------------------------------------------------
%	Executive Summary
%----------------------------------------------------------------------------------------

\chapter{Executive Summary}
This report presents a comprehensive proposal for an intelligent scale system designed to address the challenges faced by the Royal New Zealand Society for the Prevention of Cruelty to Animals (SPCA). The current weighing process causes animal stress, resulting in inaccurate measurements and occasional data loss. As Team 10, we propose our weighing scale product, which provides a fresh and modern user experience featuring hands-free automatic page transitions and an intuitive, easy-to-understand system. These features will speed the weighing process through efficient, accurate real-time data and automated animal weight tracking. By automating the weighing process, the proposed solution aims to relieve some of these challenges and allow the staff of SPCA to focus more on the health and well-being of the animals.

A stakeholder analysis was carried out, and we identified, in addition to SPCA staff, we had to take into consideration local governments, communities, and the engineers involved in this project. This helped broaden our understanding of the end-users and their interests, which was essentially an accurate and easy-to-use scale system that incorporated a wide array of data privacy, environmental, and ethical considerations. One important consideration was the values of Te Tiriti O Waitangi. It is paramount to protect the rights of Māori culture and values and the importance of establishing equitable access and use by Māori communities. Another aspect was data sovereignty; with Te Tiriti o Waitangi values, we understand the importance of protecting the data generated from Māori-owned farms and vet practices. Another aspect was ensuring that no symbols and language were used in a way that may be deemed cultural appropriation and cause offense.

Our hardware implementation consists of our circuit on PCB, Raspberry Pico W, and a 4x AA battery pack. When designing the PCB, we considered environmental and sustainability considerations to ensure we minimize our carbon impact by selective resource procurement and evaluating materials used following RoHS compliance to ensure the product's durability and recyclability. The PCB design is a small form factor to reduce E-waste, which modular components for the Pico W for easy maintenance and replacement. We addressed energy consumption by implementing power-saving features within our firmware which turned off the power to the circuit when not in use. The circuit design contains a linear voltage regulator that accounts for voltage drops from the battery pack. A Wheatstone Bridge was utilized to ensure the voltage levels from the weight scale’s four strain gauges are suitable for processing, allowing our circuit only to monitor a single signal reducing additional parts required and ultimately more cost-effective. The instrumentational amplifiers and filters improved our signal's accuracy by removing noise and adding a voltage offset to ensure the output was not negative. To provide our user’s continuous system feedback, four LEDs are on the PCB to indicate the status of Power, Wi-Fi, Stability, and Tare. To ensure the accuracy of our design decision, validation through simulation software (LTspice) was carried out, as well as experimental measurements using real weights ranging from 5 kg to 25 kg.

Our embedded software implementation consists of three key sections: a finite state machine (FSM), ADC conversion and stability check, and TCP/IP/HTTP networking stack. For the Raspberry Pico W module to receive analogue inputs from the scale, our ADC component scales and converts the ADC inputs to voltages which can be utilized by our FSM. The FSM consists of six states with the appropriate LEDs mapped to states that provide feedback to the user. The data inputs to our system occur during the ‘idle’ state where polling for weight scale data occurs, and to ensure stable and accurate weight data is collected - two different sample cycles are collected and compared. The Tare button is mapped to an Interrupt Service Routine (ISR) that allows for the Tare functionality. The frontend software architecture includes a website developed using React and Vite, which provides a fast development environment. Multiple interconnected pages were implemented, which included: a user login page, navigation bar, dashboard, chatroom, and weight tracking, additional features including notifications, and admin-specific pages such as user management. The backend server utilized a MongoDB database in tandem with Firebase authentication. MongoDB was used for the database due to its flexibility, scalability, and powerful querying capabilities. The integration of Firebase tools to our backend provided powerful security features when handling user account data which was a crucial consideration in our design.

Overall, the intelligent scale system represents a significant step towards improving animal care and welfare in New Zealand. The proposal combines technological advancements, sustainability considerations, and cultural respect to provide an efficient and automated solution for weighing animals, reducing stress, and allowing staff to prioritize animal well-being, ultimately supporting the SPCA's mission of providing optimal care for animals.

