%----------------------------------------------------------------------------------------
%	Conclusion
%----------------------------------------------------------------------------------------

\chapter{Conclusion}

In conclusion, our project aimed to develop an intelligent scale system to streamline the process of weighing dogs for SPCA. By focusing on the pet associations and franchises, we aimed to address their need for rapid and increased number of weighings of animals in a short period of time. The solution has the potential to benefit not only the SPCA but also other organisations, such as Pet Rescue and Pet Refuge, as well as the government's animal management sector.

Throughout the project, we considered various factors to ensure the success and sustainability of the intelligent scale system. We took into account the customer sector, market value, and the increasing number of animals requiring care. Furthermore, we acknowledged the importance of upholding Te Tiriti o Waitangi principles, particularly data sovereignty related to Maori culture and values.

Sustainability and environmental considerations were also incorporated into the design of our prototype. We aimed to minimise the environmental impact by using sustainable materials where possible, optimising energy consumption, ensuring durability and easy maintenance, and planning for proper disposal or recycling of the product at its end-of-life stage.

From a more technical perspective, the Finite State Machine (FSM) effectively controlled the main operation of the scale, ensuring smooth transitions between different states. GPIO port mapping facilitated user interaction through LEDs, providing clear indications of the system's status. The ADC reading and averaging methodology improved the accuracy and stability of weight measurements. The embedded TCP/IP and HTTP stack implementation allowed for seamless communication with the backend software system. By utilising the lwIP library, we successfully sent HTTPS POST requests containing measurement and identification data to the backend server. MongoDB was used as the database, providing flexibility, scalability, and efficient data storage.

In summary, our intelligent scale system prototype has fulfilled the requirements outlined by the SPCA and other stakeholders. It provides an efficient, streamlined, and automated solution for weighing dogs, reducing stress on animals and allowing staff to focus on their well-being. The system's user-friendly interface and online data storage facilitate easy access to recorded data.

In the long term, we aim to achieve profitability by expanding the system's implementation to other pet associations and franchises, conducting a thorough TAM SAM SOM analysis, and continuously improving the user experience based on customer feedback.

Overall, the intelligent scale system represents a significant step towards improving animal care in New Zealand, while also incorporating sustainability, cultural considerations, and technological advancements.

