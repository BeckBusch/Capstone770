%----------------------------------------------------------------------------------------
%	Analysis
%----------------------------------------------------------------------------------------

\begin{itemize}

\item\textbf{Risk: A student might be unavailable to complete their assigned workload in the given time frame}
\\
\\ Potential cause: Assignments from other courses might have a similar due dates as one of the deliverables of this project.
\\
\\ Example: Another assignment from a different class is due on the same day as one of the deliverables of this project. The student concludes the other assignment has a higher weight in the final result and finds it more challenging, leading them to spend more time on the other assignment. As a result, the work required for this project is rushed and not completed to the expected standard of quality.
\\
\\ Potential mitigation: Students in situations such as the example above should talk to the rest of the team as soon as they notice more than one assignment is due on the same day. With this information early on, the team can re-allocate some of the workload between the other team members. Meanwhile, the student can break down their allocated tasks and set intermediate deadlines for each task to help with time management.
\\
\\ Monitoring plan: The student will let the team know the intermediate deadlines they set via the messenger group chat and the leader (Francis) is responsible for checking whether these deadlines are met by reminding them on the day of the deadline, once again via messenger. If one of these intermediate deadlines is missed, a team meeting will be held to rearrange the workload around the given timeline.
\\
\\

\item\textbf{Risk: The software written and the hardware built does not integrate well}
\\
\\ Potential cause: Team members have different understanding with respect to project requirements.
\\
\\ Example: Since the project has both hardware and software components, the two components must be able to work together seamlessly for overall success. If software students are writing code for a feature that is not implemented in hardware, they will waste valuable time and may affect other project development activities.
\\
\\ Potential mitigation: Project requirements and implementation details will be discussed and recorded in a Google document by the team’s minute taker (Beck). A Gantt chart will also be produced during this discussion, which marks the important milestones of the project. Through this meeting, all team members must arrive at a mutual understanding of the project before any technical work is done. Understanding of the project scope includes how often data is sent from the hardware to software as well as the product’s other features.
\\
\\ Monitoring plan: The Gantt chart will be used to track the progress of development. Every time a major task is accomplished, the leader (Francis) will schedule a time for the whole team to get together using a when2meet to test if the hardware and software are working well, ensuring the project is on the right track.
\\
\\

\item\textbf{Risk: Communication between team members break down}
\\
\\ Potential cause: Members are busy with other courses and commitments.
\\
\\ Example: Everyone in the team is busy with other courses and commitments but do not communicate their situations to the rest of the team. With no regular communication and clarification, it is difficult to manage everyone’s progress with each task and consequently, the overall group’s progress.
\\
\\ Risk mitigation: The team leader (Francis) will remind team members of important dates and meeting times at least a week in advance. If there is no response from members, they will contact them individually and if this issue continues to persist, they will go to the mentor (Nimral) for advice. Meetings will also be held every Thursday during the given tutorial time which all members are required to attend. The leader will also bring this issue up and the team will use that meeting to discuss how communication within the team can be improved to be more effective.
\\
\\ Monitoring plan: Check team communication channels regularly and reply to any questions other team members have asked or help give feedback for improvement. Even for short notices or announcements, other team members will indicate that they have read by reacting to the messages. To prevent any important messages being missed in the communication channels, these notes will be pinned to ensure they are not lost. All team members must attend the Thursday meetings so that there is a set time that the team can use to communicate and work collaboratively.
\\
\\


\item\textbf{Risk: Merge issues when committing code to the repository}
\\
\\ Potential cause: Different people working on the same branch can create merge conflicts when working on the same files.
\\
\\ Example: Two members are working on the same branch, on the same file. One member commits their changes and pushes it to the branch. The other member is unaware of this new commit and pushes their changes without the updated changes, resulting in a merge conflict.
\\
\\ Potential mitigation: Create different branches for different parts of the project. Members will not work on the same file on the same branch. Use merge requests when work is ready to be reviewed and require at least two approvals before it can be merged to another branch. The local branch will be regularly updated by pulling the most recent version of that branch on GitHub to reduce the number of conflicts that arise when merging.
\\
\\ Monitoring plan: Discuss during team meetings and/or continuously communicate through established channels to ensure multiple people are not working on the same features at the same time.
\\
\\

\item\textbf{Risk: Team member struggling on a certain task}
\\
\\ Potential cause: Team members lack the necessary knowledge to finish their task as it is a new project without any necessary guidance.
\\
\\ Example: Computer Systems student hasn't encountered APIs before and struggles to send readings to the database.
\\
\\ Potential mitigation: Voice concerns, questions and difficulties in weekly meetings so that others can either attempt to help or seek out others to help. If the student requires urgent help and/or advice, they will use the messenger group chat to ask for help. If there is a member that can help the student, they will contact them personally as well as alerting the group that they will be working together. The student will also do further personal study, either by attending tutorials (if any) or online resources.
\\
\\ Monitoring plan:
Commit to meeting up weekly at the Thursday meetings and going around the group commentating on how they fared in the week. All members will express any challenges they faced during the week and how they have, or how they plan to overcome them. Minute taker records what each member is working on at the time of each meeting and if a particular member is helping another, make note to ensure everyone’s workload is relatively even.
\\
\\

\item\textbf{Risk: Development in either hardware/software falls behind}
\\
\\ Potential cause: Implementation of some aspects may be more difficult than intended or team members may struggle to implement some features according to schedule.
\\
\\ Example: Embedded components are not implemented and running by the time the electrical and software components are completed.
\\
\\ Potential mitigation: Revise design with the team and check if the implementation is feasible to complete in given time frame. Get assistance from other team members that are either on/ahead of schedule. Speak with mentor (Nirmal) for non-technical aspects and reach out to corresponding lecturer for technical support.
\\
\\ Monitoring plan: Weekly meetings with all team members in person to ensure everything is progressing as intended and provide any necessary intervention if required. Keep up to date in communication channels, giving brief updates via messenger group chat and log progress with the Gantt chart.
\\
\\

\item\textbf{Risk: Lost code due to malfunction}
\\
\\ Potential cause: Student’s computer breaks down before they can commit and push their local changes.
\\
\\ Example: A software student works on the project from their home computer. All their work is on a local branch when the computer breaks down before they can push these changes causing their work to be lost. Because none of the work had been pushed to the remote branch, no one else had access and the entire code they had written was lost.
\\
\\ Potential Mitigation: Commit regularly and push local changes every time a milestone, or intermediate step is achieved. This will drastically reduce the chances of lost code due to malfunction as once the code has been pushed remotely, even in the case a malfunction occurs, other users and the corresponding student can still access the code on the remote branch.
\\
\\ Monitoring Plan: The leader (Francis) will check commit records. If a particular student gives updates on their progress but there are no commits to support it, they will contact the student to ensure their progress is being recorded and pushed to the remote branch in the case that a malfunction occurs.
\\
\\

\item\textbf{Risk: Supply chain delays}
\\
\\ Potential cause: Component shortages, global events such as natural disasters that delay or restrict shipping from overseas.
\\
\\ Example: Electrical components that have been ordered online are taking too long to arrive due to shipping delays.
\\
\\ Potential Mitigation: Ensure that as many components as possible are sourced from the university and only order online from trusted suppliers. If required components are being supplied online, ensure there is enough time for it to be delivered and track status of delivery daily. If delivery periods seem unrealistic, all members will contribute to looking for alternative options.
\\
\\ Monitoring plan: The leader (Francis) will update team members on the delivery statuses. They will also be responsible for ensuring alternative options are found, if required by reminding members daily of this urgent situation.
\\
\\

\end{itemize}